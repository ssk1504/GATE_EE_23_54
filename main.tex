\documentclass[journal,12pt,twocolumn]{IEEEtran}
\usepackage{amsmath,amssymb,amsfonts,amsthm}
\usepackage{txfonts}
\usepackage{tkz-euclide}
\usepackage{listings}
\usepackage{gvv}
\usepackage[latin1]{inputenc}
\usepackage{adjustbox}
\usepackage{array}
\usepackage{tabularx}
\usepackage{pgf}
\usepackage{lmodern}
\usepackage{circuitikz}
\usepackage{tikz}
\usepackage{graphicx}

\begin{document}
\bibliographystyle{IEEEtran}

\vspace{3cm}

\title{}
\author{EE23BTECH11054 -  Sai Krishna Shanigarapu$^{*}$
}
\maketitle
\newpage
\bigskip

% \renewcommand{\thefigure}{\theenumi}
% \renewcommand{\thetable}{\theenumi}

\section*{Gate EE 2023}
54. \hspace{2pt}The circuit shown in the figure is initially in the steady state with the switch K in open condition and $\overline{K}$ in closed condition. The switch K is closed and $\overline{K}$ is opened simultaneously at the instant $t = t_1$, where $t_1 > 0$. The minimum value of $t_1$ in milliseconds such that there is no transient in the voltage across the 100 $\mu F$ capacitor, is \rule{1cm}{0.15mm} (Round off to 2 decimal places).

\begin{figure}[h!]
  \centering
  \resizebox{0.8\columnwidth}{!}{    \begin{circuitikz}[american]
        \draw (0,7) to [R=10$\Omega$] (0,2) to [short] (3,2) to [isource, l=\scriptsize{$\sin\brak{1000t}$}] (3,7) to [short] (0,7);
        \draw (3,2) to [short] (5,2) to [short] (5,0) to [R=$10\Omega$] (7.5,0) to [battery2 = 5V] (10,0) to [short] (10,2) to [curved capacitor=100$\mu$F, invert] (10,7) to [short] (5,7) to [short] (3,7);
        \draw (5,2) to [short] (7, 2) to[ospst=$\overline{K}$] ++(1,0);
        \draw (5,5) to [short] (5,2);
        \draw (10,2) to [short] (8,2);
        \draw (5,7) to [short] (5,6) to[cspst=K] ++(0,-1) ;
\end{circuitikz}
}
\end{figure}

\solution

Case(i) Switch K is open and $\overline{K}$ is closed.
\begin{figure}[h!]
  \centering
  \resizebox{0.50\columnwidth}{!}{    \begin{circuitikz}[american]
        \draw (0,7) to [R=10$\Omega$] (0,2) to [short] (3,2) to [isource, l=$1\angle 0^\circ$A] (3,7) to [short] (0,7);
        \draw (7,7) to [short] (7,6.5) to [curved capacitor=100$\mu$F, v=$V_C$] (7,2);
        \draw (3,7) to [short, -*, i=$i_c$] (7,7);
        \draw (3,2) to [short] (7,2);
\end{circuitikz}
}
\end{figure}

\begin{align}
    X_c &= -10j
\end{align}
Using Current divider rule,
\begin{align}
    i_c &= \frac{10}{10 + X_c} 1 \angle 0^\circ\\
    &= \frac{1\angle 0^\circ}{1-j}\\
    %V_c &= i_c X_c\\
    %&= 7.07\angle{-45^\circ} V\\
    \implies V_c &= 7.07\sin \brak{1000t - 45^\circ}V
\end{align}

%\begin{table}[h!]
%    \setlength{\arrayrulewidth}{0.3mm}
\setlength{\tabcolsep}{15pt}
\renewcommand{\arraystretch}{1.4}

\resizebox{10cm}{!}{
\begin{tabular}{|c|c|c|}
\hline

Symbol & description & value\\
\hline
$V_C\brak{\infty}$ & Voltage across capacitor after long time & $5V$\\
\hline
$\tau$ & Time constant & 1 msec\\
\hline
%$V_c\brak{t}$ & Voltage across capacitor at time t & $5 + \brak{7.07\sin \brak{100t - 45^\circ}-5}e^{-\brak{t-t_1}/ \tau}$\\
%\hline
R & Resistance & 10$\Omega$\\
\hline
C & capacitance & 100$\mu$F\\
\hline
f & frequency of the current source & $\frac{500}{\pi}$\\
\hline


\end{tabular}
}

%    \caption{Case1 Parameters}
%    \label{tab:Gate.ee.54.1}
%\end{table}

\newpage
Case(ii) Switch K is closed and $\overline{K}$ is open.

\begin{figure}[h!]
  \centering
  \resizebox{0.55\columnwidth}{!}{%\begin{circuitikz}[american]
        %\draw (0,0) to [R=10$\Omega$] (1.5,0) to [battery2=5V] (4,0);
        %\draw (0,0) to [short] (0,3) to [short] (4,3) to [C=100$\mu$F, %v=$V_c\brak{\infty}$] (4,0) ;
%\end{circuitikz}

\begin{circuitikz}[american]
    \draw (0,0) to [R=10$\Omega$] (1.5,0) to [battery2=$\frac{5}{s}$] (4,0);
    \draw[<-, thick] (1.5,1) arc (-90:90:0.5) node[midway, left] {$I(s)$};
    \draw (0,0) to [short] (0,3) to [short] (4,3) to [C=$\frac{10^4}{s}$, v=$V_c\brak{s}$] (4,0);
\end{circuitikz}

}
\end{figure}

\begin{table}[ht]
    \setlength{\arrayrulewidth}{0.3mm}
\setlength{\tabcolsep}{15pt}
\renewcommand{\arraystretch}{1.4}

\resizebox{10cm}{!}{
\begin{tabular}{|c|c|c|}
\hline

Symbol & description & value\\
\hline
$V_C\brak{\infty}$ & Voltage across capacitor after long time & $5V$\\
\hline
$\tau$ & Time constant & 1 msec\\
\hline
%$V_c\brak{t}$ & Voltage across capacitor at time t & $5 + \brak{7.07\sin \brak{100t - 45^\circ}-5}e^{-\brak{t-t_1}/ \tau}$\\
%\hline
R & Resistance & 10$\Omega$\\
\hline
C & capacitance & 100$\mu$F\\
\hline
f & frequency of the current source & $\frac{500}{\pi}$\\
\hline


\end{tabular}
}

    \caption{Parameters}
    \label{tab:Gate.ee.54.1}
\end{table}

%From Table \ref{tab:Gate.ee.54.1} and Table \ref{tab:Gate.ee.54.2},
The voltage across capacitor at time t is given as,

\begin{align}
    V_c\brak{t} &= V_c\brak{\infty}+\brak{V_c\brak{0}-V_c\brak{\infty}}e^{-t/ \tau}\\
    \implies V_c\brak{t} &= 5 + \brak{7.07\sin \brak{1000t - 45^\circ}-5}e^{-t/ \tau}
\end{align}

For transient free voltage,
\begin{align}
    7.07\sin \brak{1000t_1 - 45^\circ} &= 5\\
    1000t_1 - \frac{\pi}{4} &= \frac{5}{7.07}\\
    \implies t_1 &\approx 1.57 \text{msec}
\end{align}

\begin{table}[ht]
    \setlength{\arrayrulewidth}{0.3mm}
\setlength{\tabcolsep}{15pt}
\renewcommand{\arraystretch}{1.4}

\resizebox{10cm}{!}{
\begin{tabular}{|c|c|c|}
\hline

Symbol & Description & Formula\\
\hline
$\tau$ & Time constant & RC\\
\hline
%$V_c\brak{t}$ & Voltage across capacitor at time t &  $V_c\brak{\infty}+\brak{V_c\brak{0}-V_c\brak{\infty}}e^{-t/ \tau}$\\
$X_c$ & Capacitive reactance & $\frac{1}{2\pi f C}$\\
\hline
$V_c$ & Voltage across capacitor & $i_c X_c$\\
\hline


\end{tabular}



}

    \caption{Formulae}
    \label{tab:Gate.ee.54.2}
\end{table}


\end{document}
